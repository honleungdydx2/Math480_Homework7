\documentclass[11pt,letter]{article}
\usepackage{latexsym}
\usepackage{mathrsfs}
\usepackage{amsmath}
\usepackage{amssymb}
\usepackage{amsthm}
\usepackage{multicol}
\usepackage{enumerate}

%numbering theorems
\newtheorem{thm}{Theorem}
\newtheorem{defi}[thm]{Definition}
\newtheorem{axi}[thm]{Axiom}
\newtheorem{prop}[thm]{Proposition}
\newtheorem{rem}[thm]{Remark}
\newtheorem{cor}[thm]{Corollary}
\newtheorem{lem}[thm]{Lemma}
\newtheorem{eg}[thm]{Example}
\numberwithin{equation}{section}

% defining new commands
\newcommand{\dis}{\displaystyle}

\newcommand{\N}{\mathbb{N}}
\newcommand{\R}{\mathbb{R}}
\newcommand{\C}{\mathbb{C}}
\newcommand{\Z}{\mathbb{Z}}
\newcommand{\Q}{\mathbb{Q}}
\newcommand{\Ba}{\mathbb{B}}
\newcommand{\D}{\mathbb{D}}
\newcommand{\Sp}{\mathbb{S}}
\newcommand{\A}{\mathcal{A}}
\newcommand{\B}{\mathcal{B}}
\newcommand{\E}{\mathcal{E}}
\newcommand{\W}{\mathcal{W}}
\newcommand{\ve}{\varepsilon}
\newcommand{\vp}{\varphi}
\newcommand{\mc}{\mathcal}
\newcommand{\ra}{\rightarrow}
\newcommand{\wt}{\widetilde}
\newcommand{\wh}{\widehat}
\newcommand{\ul}{\underline}
\newcommand{\ol}{\overline}
\newcommand{\triq}{\triangleq}


\begin{document}
\title{Fermat-Toricelli Problem: SDP formulation}
\author{Hon Leung Lee}
\date{\today}
\maketitle

\section{Problem formulation}

We describe here the idea of SDP formulation in \cite{NPS}. Our problem is to minimize 
over $x\in \R^2$, $\dis F(x) = \sum_{j=1}^n \eta_j \|x-a_j\|$ where all $\eta_j$ is positive and $a_j \in \R^2$. 
Let $x$ and $a_j$ have coordinates $(x,y)$ and $(a_j,b_j)$ respectively (some abuse of notation here).
Then the objective function  becomes
$$
F(x,y) = \sum_{j=1}^ n \eta_j \sqrt{(x-a_j)^2 + (y- b_j)^2}.
$$
Let the minimization problem be $(\mathcal{P})$. \\

Introduce slack variables $d_j = \sqrt{(x-a_j)^2 + (y- b_j)^2}$ for all $j$. Then 
$d_j^2  = (x-a_j)^2 + (y- b_j)^2 $. One can use this to show that:

\begin{thm}
The problem $(\mathcal{P})$ is equivalent to the semidefinite program (SDP) $(\mathcal{P}_2)$:
\begin{align*}
& \text{minimize } F(x) = \sum_{j=1}^n \eta_j d_j \text{ subject to }
\begin{bmatrix}
d_j + x - a_j & y - b_j \\
y - b_j & d_j - x + a_j
\end{bmatrix}
\succeq 0 \\
&  \text{ for all }j = 1,\cdots,n
\end{align*}
(Here $A \succeq 0$ means the matrix $A$ is positive semidefinite; Notice 
the decision variables of $(\mathcal{P}_2)$ are $d_1,\cdots,d_n,x,y$).
\end{thm}


\begin{proof}
For each $j$ let $A_j = \begin{bmatrix}
d_j + x - a_j & y - b_j \\
y - b_j & d_j - x + a_j
\end{bmatrix}$. If\\ $d_j = \sqrt{(x-a_j)^2 + (y-b_j)^2}$ then all principal minors of 
$A_j$ are nonnegative:
$d_j \geq |x-a_j|$ and $\dis \det (A)= d_j^2 - (x-a_j)^2 - (y-b_j)^2 = 0$. Thus $A_j\succeq 0$. So feasibility of ($\mathcal{P}$) implies 
the feasibility of ($\mathcal{P}_2$), hence the minimum value in ($\mathcal{P}_2$) cannot exceed that of ($\mathcal{P}$). 
On the other hand if  $A_j \succeq 0$, then all the eigenvalues of $A_j$ are nonnegative. By assessing the eigenvalues using quadratic formula one sees $d_j \geq \sqrt{(x-a_j)^2 + (y-b_j)^2} \geq$ the minimum value of $(\mathcal{P})$. 
This completes the other direction that the minimum value of $(\mathcal{P})$ is at most that of $(\mathcal{P}_2)$, and 
so the conclusion follows.
\end{proof}

The above SDP ($\mathcal{P}_2$) can be implemented using the SDP solver in \verb+cvxopt+ package in Python. 

\section{Discussion}

We already implemented the following baby example using the \verb+cvxopt+ package.

\begin{eg}
If $(a_0, b_0) =  (0,0)$, $(a_1,b_1) = (1,0)$ and $\dis (a_2,b_2) = \left( \frac{1}{2}, \frac{\sqrt{3}}{2} \right)$ then 
the Fermat point has to be the centroid $\dis \left(\frac{1}{2}, \frac{1}{2\sqrt{3}}\right)$, whence the minimum distance is $\sqrt{3}$. 

\end{eg}

The solver provided more information which we are also interested. For the above example it takes 6 iterations. The CPU time is 6.73 ms per loop. Also the convergence looks superlinear. This is nice.

\section{Next step}

\begin{enumerate}
\item
Implement the SDP for general $n$ points in $\R^2$ and general weights. 
\item
Perform numerical experiments through simple sampling. Fix $n  = 5$ (or any nice integer). Generate 5 points in $\R^2$ using Gaussian distribution, and generate 500 such data. Run the algorithm and analyze the results. 
\end{enumerate}



\begin{thebibliography}{9999}



\bibitem{NPS}
{\sc Nie, J., Parrilo, P. A..Sturmfels, B. }: 
{\it Semidefinite Representation of the k-Ellipse. In Algorithms in Algebraic Geometry}.
Springer New York (2008),  117-132. 















\end{thebibliography}



\end{document} 